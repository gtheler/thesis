\makeatletter
\newlength{\manejeparacentrar}
\setlength{\manejeparacentrar}{\Gm@bindingoffset}
\makeatother

\makeatletter
\def\maketitle{%
\thispagestyle{empty}
\addtolength{\oddsidemargin}{-0.5\manejeparacentrar}
 \begin{center}

  \parbox{0.75\linewidth}{\centering\large{\textsc{Tesis~de~la~Carrera~de\\Doctorado~en~Ingeniería~Nuclear}}}

  \vspace{3.5cm}

  \parbox{0.5\linewidth}{\centering\large{\textbf{\@title}}}

  \vspace{1.2cm}

   \begin{center}
    \makebox{\@author} \\
    \makebox{\textbf{Doctorando}}
   \end{center}

  \vspace{2cm}

  \begin{minipage}{0.45\linewidth}
   \begin{center}
    \makebox{Dr.~Fabián~J.~Bonetto} \\
    \makebox{\textbf{Co-director}}
   \end{center}
  \end{minipage}
  \begin{minipage}{0.45\linewidth}
   \begin{center}
    \makebox{Dr.~Alejandro~Clausse} \\
    \makebox{\textbf{Co-director}}
   \end{center}
  \end{minipage}

  \vspace{\fill}

  San Carlos de Bariloche\\
  \@fecha

  \vspace{2cm}

  Instituto Balseiro\\
  Universidad Nacional de Cuyo\\
  Comisión Nacional de Energía Atómica\\
  Argentina

\end{center}
\cleardoublepage
\addtolength{\oddsidemargin}{0.5\manejeparacentrar}
}

\def\englishtitle#1{\def\@englishtitle{#1}}
\def\fecha#1{\def\@fecha{#1}}


% -=-=-=-=-=-=-=-=-=-=-=-=-=-=-=-=-=-=-=-=-=-=-=-=
%  redefinicion del formato de los capitulos
%  borrar o cambiar a gusto
% -=-=-=-=-=-=-=-=-=-=-=-=-=-=-=-=-=-=-=-=-=-=-=-=
\makeatletter

\newlength{\izquierda}
\newlength{\capitulo}
\newlength{\derecha}
% 
% \def\@makechapterhead#1{%
%   {
%    % horrible maneje %
%    \setlength{\izquierda}{0.75cm}
% %    \settowidth{\capitulo}{\mbox{\,\large{\textsc{\@chapapp~\capituloromano}}\,}}
%    \settowidth{\capitulo}{\mbox{\,\large{\textsc{\@chapapp~\thechapter}}\,}}
% 
%    \setlength{\derecha}{\textwidth}
%    \addtolength{\derecha}{-\capitulo}
%    \addtolength{\derecha}{-\izquierda}
% 
% 
%    \noindent \parbox{\textwidth}{\rule[4pt]{\izquierda}{0.5pt}%
% %    \mbox{\,\large{\textsc{\@chapapp~\capituloromano}}\,}%
%    \mbox{\,\large{\textsc{\@chapapp~\thechapter}}\,}%
%    \rule[4pt]{\derecha}{0.5pt}}%
% 
%    \vspace{1.2cm}
% 
% % alineado a la izquierda
% %  \hspace{1.5cm}\parbox[c]{12cm}{\linespread{1.1}\raggedright\hspace{-0.5cm}\huge{\textsf{\bfseries{#1}}}}
% 
% %centrado
%    \linespread{1.1}\centering\huge{\textsf{\bfseries{#1}}}
% 
%    \vspace{2cm}
% 
% %  \noindent \rule{\textwidth}{0.5pt}
%    \nobreak
%    \vspace{0.25cm}
% 
%    % nada de headers
%    \thispagestyle{empty}
% 
%   }%
% }
% 
% \def\@makeschapterhead#1{%
%   {
%    ~
%    \vspace{1.5cm}
% 
%    \noindent \huge{\textsf{\bfseries{#1}}}
% 
%    \vspace{1.5cm}
% 
%    \thispagestyle{empty}
%   }%
% }

% \def\@part[#1]#2{%
%     \ifnum \c@secnumdepth >-2\relax
%       \refstepcounter{part}%
%       \addcontentsline{toc}{part}{\thepart\hspace{1em}#1}%
%     \else
%       \addcontentsline{toc}{part}{#1}%
%     \fi
%     \markboth{}{}%
%     {\vspace{3.5cm}
%      \centering
%      \interlinepenalty \@M
%      \normalfont
%      \ifnum \c@secnumdepth >-2\relax
%        \textsf\huge\bfseries  \partname\nobreakspace\thepart
%        \par
%        \vskip 20\p@
%      \fi
%      \thispagestyle{empty}
%      \Huge \bfseries #2\par}%
%     \vfill
%     \hfill\includegraphics{part\thepart}
%     \vspace{1.5cm}
%     \@endpart}


\makeatother

\newenvironment{chapterquote}[1][0.7\linewidth]{
\vspace{0.5 cm plus 0.25cm minus 0.25cm}
\hspace{\fill}\begin{minipage}{#1}
  \begin{flushright}
    \bgroup
    \par
    \sffamily\selectfont%
}
{%
    \par\egroup
  \end{flushright}
\end{minipage}
\par\vspace{2cm plus 0.5cm minus 0.5cm}%
}



% -=-=-=-=-=-=-=-=-=-=-=-=-=-=-=-=-=-=-=-=-=-=-=-=
% definiciones y renews especificos para la tesis
% -=-=-=-=-=-=-=-=-=-=-=-=-=-=-=-=-=-=-=-=-=-=-=-=
% \newcommand{\omegaversor}{\hat{\symbf{\Omega}}}
% \newcommand{\omegaprimaversor}{\hat{\symbf{\Omega}}^\prime}

% los vectores son en bold, no con una flecha maraca arriba
% \renewcommand{\vec}[1]{\ensuremath\mathbf{#1}}
% las matrices son asi
% \newcommand{\mat}[1]{\ensuremath\mathsf{#1}}

% para isotopos, esta bueno esta definicion y hacer
% \nucl{235}{92}{U} o directamente \nucl{235}{}{U}
% \newcommand{\nucl}[3]{
% \ensuremath{
% \phantom{\ensuremath{^{#1}_{#2}}}
% \llap{\ensuremath{^{#1}}}
% \llap{\ensuremath{_{\rule{0pt}{.75em}#2}}}
% \mbox{#3}
% }
% }
% 
